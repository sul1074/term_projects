%%% written by Karnes Kim.
%\backmatter
%\appendix
\chapter*{역자후기}
\markboth{}{}

거의 15년만에 lshort을 다시 번역하여 내게 되었다.
이 책자는 수십 년에 이르는 유구한 시간의 시련을 견뎌내며 \LaTeX{} 입문서로서 
그 성가를 쌓아왔다. 일독을 권한다. 입문자라면 예제를 중심으로 
조금 시간을 내어 공부한다면 얻는 것이 적지 않을 것이다.

지금은 각국 언어로 번역이 이루어져 CTAN에서 찾아볼 수 있지만 
한국어판은 다른 언어와 비교해도 상당히 이른 시기에 번역본이 나왔고 
CTAN에 올라가 있은 지도 오래 되었다. 그리고 그 기간 갱신이 이루어지지 
않아서 유감스럽게 생각하다가 이번에 새 판을 내게 되어 매우 흡족하다.
% 몇몇 분들이 번역을 시도한다는 소식을 들은 바가 있으나 어쩐 일인지 
% 잘 진척되지 않았던 모양이다.

두 번에 걸친 lshort 한국어판 번역 작업은 한글 \LaTeX 의 발전 과정과 
그 궤를 같이해 왔다. 2002년에는 \HLaTeX 으로, 2005년에는 dhucs로 
작업하였으니 실로 격세지감이 있다 하겠다. 이제 \koTeX 과 \XeLaTeX 이 널리 쓰이게 되어 
이런 훌륭한 결과물을 비교적 고생하지 않고 간단히 얻을 수 있게 된 것은 
특기하여야 할 일일 것이다. 한글 \LaTeX{} 개발을 위해 헌신한 모든 분들께 
감사 인사를 드린다.

이번 번역은 이 책이 실용적인 한글 입문서가 되어야 한다는 관점에서 
작업이 이루어졌다. 몇 가지 지난 번역본의 역어를 포기하지는 않았지만 
지나친 순정주의를 지양하고 쉽게 읽히면서 \LaTeX 을 배우는 데 실제 
도움이 될 만하게 만들려고 노력하였다. 

역자들은 세 가지 방식으로 본문에 개입하였는데 (1) 방대한 역주를 달아서 
독자의 이해를 도우려 하였으며, (2) 필요하다면 본문에 추가해 넣는 것도 마다하지 않았다.
그러나 본문의 추가는 최소한으로 억제하려 하였고 원문을
충분히 존중하면서 훼손하지 않으려고 애썼다. (3) 한국어 문서 
작성과 관련된 두세 개의 소절을 추가하였다. 이 세 가지 역자의 개입은 
보는 즉시 바로 알아볼 수 있도록 표지를 붙여두었다.

한국어판은 \textsf{oblivoir} 클래스를 바탕으로 작성하였으며 
원본의 이곳저곳에 있는 명령, 환경, 설정의 많은 부분을 재구현하였다. 

초고본을 KTUG 게시판에 공개하고 의견을 구하는 과정에 참여해주신 분들께 감사드린다.
좋은 문서가 되는 것은 독자의 기여가 없이 불가능한 것이라고 믿는다. 
특별히 몇 번에 걸쳐 읽으면서 수많은 오자를 꼼꼼하게 잡아내고 좋은 표현을 제안해준
이주호, 이호재, 윤석천 제씨의 노고에 깊은 감사를 드린다.

우리가 들인 시간과 노력이 \LaTeX 에 입문하는 분들께 조금이라도 얻게 해주는 바가 있다면 
큰 기쁨이겠다. 
최신판은 \url{https://github.com/KoreanTUG/lshort-ko}에서 볼 수 있을 것이며
어떤 형태의 기여와 조언도 환영한다.

{\flushright 2019년 2월 \\
김강수\cntrdot 조인성 \texttt{<ischo@ktug.org>} \par
}


\subsection*{한글 폰트에 관하여}

이 문서의 배포판에서 영문자는 원본과 동일하게 Latin Modern 폰트를 이용하였다.
한글은 KoPub World 폰트를 사용하였는데 이는 한국출판인회의에서 전자출판진흥사업의 일부로 
무료 공개한 서체이다. \url{http://www.kopus.org/biz/electronic/font.aspx}에서 
얻을 수 있다.

\section*{2005년판 4.17의 역자후기}

KTUG이 생기기 전에 필자의 개인 홈페이지에서 처음 번역이 이루어진 lshort-kr은
그 후 CTAN에 등록되어 우리글로 된 \LaTeX\ 입문서로서 많은 분들에게 읽혀왔다.
지금 돌이켜보면 당시 H\LaTeX 을 기반으로 번역 작업을 하면서 고생했던 것은
영문을 옮기는 것도 옮기는 것이지만 \LaTeX 에서 한글을 구현하는 것 자체가
쉽지 않았던 면이 있었다.

KTUG이 이루어놓은 놀라운 업적들이 이 번역본에 고스란히 담겨 있다. 우선 
최종 출력물인 PDF의 품위 자체가 달라졌다. 한글 책갈피, 텍스트의 추출 $\cdot$
검색, 하이퍼링크 등, 3.20판을 번역할 당시에는 잘 상상하기 어렵던 일들이
너무나 손쉽게 가능해졌다. 또한 유니코드 기반의 한글로 이행함으로써
더이상 H\LaTeX 의 EUC-KR 한계를 걱정하지 않아도 되는 행복한 상황에서
4.17판을 번역하게 되었다. 이 한글판 lshort-kr은 \textsf{unicode/dhucs} 패키지를
이용하여 한글을 구현하였다.

기술적 뒷받침이 이루어졌으므로 이제는 내용의 질을 제고할 때라고 생각한다.
이번 번역도 자원한 분들과 함께 공동작업으로 이루어졌다. 영어에 능한
분들과 언어학을 전공하신 분까지 번역에 합류함으로써 이제는 번역의 질에
있어서도 부끄럽지 않을 정도가 되었다고 생각한다.

이번에 공동번역자들이 합의한 번역 원칙은 다음과 같다.
\begin{itemize}
\item 영문을 그대로 옮겨놓는 번역을 피하고 실제 \LaTeX 에 입문하는 초보자들이 쉽게 이해하고 적용할 수 있도록
    문장을 완전히 새로 쓴다. 필요하다면 내용을 보충하거나 생략할 수 있다. 중요한 것은 입문서로서 이 책의
    전달력이지 원문의 충실한 재현에 있지 않다. 즉, 번역문에서 어떤 영문 문장의 기미도 발견할 수 없도록,
    아름다운 우리말로 이루어진 문서를 만드는 것이 목표이다. (우리말식 용어를 사용해야 한다는 뜻은 결코
    아님).

\item 예제는 이번에도 영문을 그대로 노출시킨다. 한글 \LaTeX 의 발전에 힘입어 예제를 한글화하는 것이 어려운
    일은 아니지만 여전히 이 책은 \LaTeX{} 사용법에 관한 책이다.
    그러나 한글 \LaTeX{} 사용환경이 현저히 다르거나 한글화와 관련한 중요한 사항이 있을 때는
    이에 대한 언급을 별도의 절이나 역주로 만들어 붙인다.

\item 용어의 통일은 중요하다. 역어의 선택은 되도록이면 우리말화하되 우리말 용어 자체를 지나치게 중시한 나머지
    도무지 알 수 없거나 머리 속에서 영어로 다시 옮겨야 이해가 되는 것을 배제하고 적당한 우리말이 없는
    경우에는 차라리 흔히 쓰이는 영어식 표현을 그대로 쓴다.
    
\end{itemize}

모든 경우에 이 원칙이 완전히 관철되지는 않았을 것이다. 그러나 번역자들이 지향하는
번역의 방향은 대체로 일치하였다고 생각한다. 

한국어판이 개선되지 않는데도 불구하고 메일링 리스트에서 삭제하지 않고
꾸준히 훌륭한 문서를 만들어 온 저자 Tobias Oetiker 씨에게 감사한다. 
원래 한국어판 번역본에 실렸던 글을 더 확장하고 보충하여 영문판에 실리도록
글을 써주신 신정식 님께도 감사드린다. 이 번역본에 새로이 수정하여 실린
``한국어 지원'' 절은 lshort의 다음 판에서 이용할 수 있도록 Oetiker 씨에게
수정본을 보낼 생각이다.
사실상 KTUG의 활동을 가능케 한 조진환 박사, KTUG에 깊은 애정을 보이시는
지도자이자 후원자이신 남상호 박사, dhucs와 dhhangul의 저자로서 한글화에
중요한 기여를 하신 김도현 교수, KTUG의 중심인 이주호 님, 이호재 님,
KTUG Collection을 가능하게 한 홍석호 님, 그밖에 KTUG과 lshort-kr에
관심과 격려를 보내주신 모든 질문자와 답변자 분들께 감사드린다. 이주호 님은
번역본 전체를 읽으면서 주교정자로서의 역할을 담당해주셨다.
특별히 교정\cntrdot 교열 과정에서 긴 교정표를 작성해주신 이상직 교수, 그리고 
번역본을 읽고 의견을 제시해주신 딸기아빠 님과 커꿈 님께도 감사드린다.
번역본이 더 읽고 이해하기 쉬워진 것은 전적으로 이 분들의 공이다.

번역이 이루어질 무렵 한글날을 맞아 한겨레신문사에서 한겨레결체를 공개한 것은
공개 한글 글꼴이 부족한 \TeX\ 공동체에 좋은 선물이었다. 이 문서는
한겨레결체를 본문 글꼴로 채용하였다.

이 번역본에 잘못이 있다면 책임은 김강수에게 있다. 공동 역자들은 오역에
대하여 책임이 없다. 초보자분으로서 이 책에서 조금이라도 얻은 것이 있다면
역자들에게 알려주기 바란다. 개선을 위한 의견, 오류의 지적도 환영한다.
KTUG 게시판이나 전자우편을 이용할 수 있다.

함께 고생한 공동 역자들의 노력에 감사하고 행운을 빌면서.

\bigskip
번역자 : 김강수, 이기황, MIKA, 샘처럼, 김지운.

\begin{flushright}
공동번역자를 대표하여\ldots\ 김강수
\end{flushright}

\section*{2002년판 3.20판의 역자후기}
%\chapter*{역자후기 (v. 3.20)}
%\newcommand\HLaTeX{한글\LaTeX}
\setcounter{table}{0}
\setcounter{footnote}{0}

이 책은 사용법이 쉽지만은 않은 \LaTeX의 입문서로 이미 정평이 있다.

당연히 \LaTeXe를 제대로 이해하고 쓰기 위해서는 \companion이 있어야 할 것이고,
좀더 고급의 독자들은 \emph{The \TeX{}book}을 보아야 하겠지만, 
논문 작성 등 일반적 용도에는 이 책이 제공하는 정도의 기능만 
숙지하더라도 충분히 자신의 목적을 달성할 수 있을 것이다.

이 책을 번역해야겠다는 생각은 오래 전부터 가지고 있었는데,
그것을 실행에 옮길 엄두를 내기가 어려웠다.
우선, 한글판 lshort가 과연 필요할 것인가도 확신하기 어려웠고(왜냐하면 어차피 이 책을 한국어로 옮긴다 하더라도 예제는 여전히 영어 예제를 쓸 수밖에 없으며, 한글 구현에 관한 사항은 이 글의 `번역'에서는 다룰 수
없었기 때문이다.), 사실 초창기 lshort는
영문판도 컴파일이 잘 되지 않는 경우가 있어서, 이것이 과연 한글로
제대로 동작할 것인지 확신할 수 없는 상태였기 때문에,
그냥 영문판을 보는 것으로 만족하고 지낸 것이 사실이다.

그러던 차에, 나의 개인 홈페이지
%\footnote{
%	\texttt{http://www.doeun.pe.kr}
%} 
게시판에서 이 문제를 제기했더니 강윤배 $\cdot$ 장대훈 님이 흔쾌히 돕겠다는 의사를 밝혀 주셨다.
이렇게 의기투합하여, 대부분의 본문을
한글로 옮기는 일을 두 분이 하고, 나는 한글\LaTeX으로 컴파일이 되도록
맞추는 일을 주로 하면서 초벌번역이 이루어졌다.
초벌번역이 끝날 무렵, 김재우 님께서는 다른 경로로 나에게 연락을 
해오셨는데, 3.1의 번역을 이미 해두신 적이 있다는 것이었다.
이렇게 전체의 번역이 이루어진 후, 내가 각 장을 다시 읽으면서
교열하고 오역을 수정하는 작업을 거쳐 마침내 한국어판 lshort를
출판(!)하게 되었다.

각 장별로 최종적으로 사용된 텍스트의 초벌 역자와 교열자는 다음 표와 같다.
\begin{table}[!htbp]
\centering%
\begin{tabular}{lll}\hline
감사의 말, 서문 & 강윤배$\cdot$장대훈 & 김강수 \\
제1장 & 김강수 & 장대훈$\cdot$김강수 \\
제2장 & 강윤배 & 장대훈$\cdot$김강수 \\
제3장 & 장대훈 & 김강수 \\
제4장 & 김재우$^*$ & 김강수 \\
제5장 & 장대훈 & 김강수 \\
참고문헌 & 장대훈 & 김강수\\ \hline
\end{tabular}\ \\
\footnotesize{*
장대훈 님의 번역본도 있었는데, 김재우 님의 텍스트를\\
주로 살리면서 장대훈 님의 번역을 참조하여 교열하였음.}
%\caption[]{각 장별 번역자와 교열자}\label{tr}
\end{table}

이 책이 \LaTeX에 입문하는 분들에게 좋은 선물이 되기를 바란다.
사실 한글로 이루어진 \TeX{} 관련서적이 거의 없다 해도 좋을 정도의 상황에서, 이 글이 가치있는 입문서 구실을 충분히 할 것으로 믿는다.

\bigskip

번역상 주의한 것은 다음과 같다.
\begin{itemize}
\item 예제들은 영문을 그냥 노출시켰다. 이렇게 한 이유는, 이 책이 \LaTeXe에 대한 설명이지 \HLaTeX에 대한 설명이 아니라는 점 때문이었다. 다시 말하면 이 예제들을 한글화했을 때, 그것은 \LaTeXe를 통해 실행되는
것이 아니라 \HLaTeX을 통해서 실행되는 것이므로, 이 책의 원래 의도와는
동떨어진 것이 된다. \LaTeX에서의 한글 사용에 대한 좋은 입문서가 나오기를 바라는 마음 간절하다. 아니 그보다, 안심하고 쓸 수 있는 한글 \TeX이 하나 있었으면 하는 생각도 든다.
\item 문장의 번역은 무엇보다도 \LaTeX{} 입문자들이 가장 잘 이해할 수 있게 하는 데 초점을 맞추었다. 필요하다면 설명을 길게 덧붙이기도 했고 몇 가지 역자에 의한 보충도 추가하였다. 이런 시도가 도움이 되기를 바란다.
\item 용어는 공동번역자들이 통일하기 위해서 여러번 시도했지만 완전히
일치하지는 못했을 수도 있을 것이다. 이 문제는 차차 고쳐가겠다.
\item 최근 CTAN의 디렉토리 구조가 바뀌면서 이 책에 나오는 URL 정보가
달라진 것이 있어서 그것을 바로잡았다.
\end{itemize}

책을 옮기는 일은 솔직히 말하면 쉽지 않았다. 그 과정에서 격려해 준 김도현 님, 이현호 님, `무식인' 님을 비롯한 모든 분들에게 특별히 감사의 말을 전한다.
초벌 번역본과 교열본을 읽고 어색한 표현을 지적해 준 \LaTeX을 전혀 모르는 나의 학생들에게도 고맙다는 인사를 전한다.
\HLaTeX의 저자인 은광희 님께 감사한다. 한글\LaTeX이 없었으면 이 글의 번역은 불가능했을 것이다.

이 번역본의 모든 책임은 김강수에게 있다. 다른 공동역자들은 오역에 대하여 책임이 없다.

\bigskip
번역자 : 김강수, 강윤배, 
장대훈, 김재우,
이재승, 현범석, 주철

\endinput
