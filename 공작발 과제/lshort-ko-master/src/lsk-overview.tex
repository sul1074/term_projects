% %%%%%%%%%%%%%%%%%%%%%%%%%%%%%%%%%%%%%%%%%%%%%%%%%%%%%%%%%%%%%%%%%
% % Contents: Who contributed to this Document
% % $Id$
% %%%%%%%%%%%%%%%%%%%%%%%%%%%%%%%%%%%%%%%%%%%%%%%%%%%%%%%%%%%%%%%%%

% % Because this introduction is the reader's first impression, I have
% % edited very heavily to try to clarify and economize the language.
% % I hope you do not mind! I always try to ask "is this word needed?"
% % in my own writing but I don't want to impose my style on you...
% % but here I think it may be more important than the rest of the book.
% % --baron

% \chapter{Preface}
\chapter{서 문}

% \LaTeX{} \cite{manual} is a typesetting system that is very
% suitable for producing scientific and mathematical documents of high
% typographical quality. It is also suitable for producing all
% sorts of other documents, from simple letters to complete books.
% \LaTeX{} uses \TeX{} \cite{texbook} as its formatting engine.
\LaTeX{} \cite{manual}은 과학 및 수학 문서를 작성하는 데 적합한 조판 시스템으로서 대단히 뛰어난 타이포그래피 품질을 얻을 수 있게 한다. 단순한 편지에서 완전한 단행본에 이르기까지 다양한 종류의 문서를 만드는 데도 적합하다.
\LaTeX 은 \TeX\ \cite{texbook}을 조판 엔진으로 사용한다.

% This short introduction describes \LaTeXe{} and should be sufficient
% for most applications of \LaTeX. Refer to~\cite{manual,companion} for
% a complete description of the \LaTeX{} system.
이 길지 않은 입문서는 \LaTeXe 에 대해 설명한다. 일반적인 \LaTeX{} 활용에 충분할 정도로 
설명할 것이다. \LaTeX 시스템에 대한 완전한 설명을 보려면 \cite{manual, companion}\를 참조하라.

% \bigskip
\bigskip
% \noindent This introduction is split into 6 chapters:
\noindent 이 안내서는 여섯 장(chapter)으로 이루어져 있다.
% \begin{description}
\begin{description}
% \item[Chapter 1] tells you about the basic structure of \LaTeXe{}
%   documents. You will also learn a bit about the history of \LaTeX{}.
%   After reading this chapter, you should have a rough understanding how
%   \LaTeX{} works.
\item[제 1 장] \LaTeXe 의 기본 구조를 설명한다. \LaTeX 의 역사에 대해서도 조금 알게 될 것이다. 이 장을 읽고 나면 \LaTeX 이 어떻게 동작하는지에 대해 어렴풋이 이해할 수 있을 것이다.
% \item[Chapter 2] goes into the details of typesetting your
%   documents. It explains most of the essential \LaTeX{} commands and
%   environments. After reading this chapter, you will be able to write
%   your first documents, with itemized lists, tables, graphics and floating bodies.
\item[제 2 장] 문서 조판의 세부사항을 다룬다. 필수적인 \LaTeX{} 명령과 환경 거의 대부분을 설명한다. 이 장을 읽고 나면 리스트 문단, 표, 그림, 떠다니는 개체 등을 포함하는 문서를 처음으로 작성할 수 있게 될 것이다.
% \item[Chapter 3] explains how to typeset formulae with \LaTeX. Many
%   examples demonstrate how to use one of \LaTeX{}'s
%   main strengths. At the end of the chapter are tables listing
%   all mathematical symbols available in \LaTeX{}.
\item[제 3 장] \LaTeX 으로 수학식을 식자하는 방법을 설명한다. \LaTeX 의 가장 강력한 기능인 수식 조판에 대해 다양한 예제를 통해 알려준다. 이 장의 끝에는 \LaTeX 으로 표현할 수 있는 수학 기호 거의 전부를 표로 정리해 두었다.
% \item[Chapter 4] explains indexes,  bibliography generation and some finer points about creating PDFs.
\item[제 4 장] 색인과 문헌 목록 만들기, 그리고 PDF 생성을 위한 미세설정에 대해서도 약간 다룬다.
% \item[Chapter 5] shows how to use \LaTeX{} for creating graphics. Instead
%   of drawing a picture with some graphics program, saving it to a file and
%   then including it into \LaTeX{}, you describe the picture and have \LaTeX{}
%   draw it for you.
\item[제 5 장] \LaTeX 에서 그림을 그리는 방법을 보여준다. 외부 프로그램으로 그림을 그리고 그것을 파일로 저장하여 문서에 불러들이는 것이 아니라 \LaTeX{} 언어로 그림을 표현하는 방법을 설명한다.
% \item[Chapter 6] contains some potentially dangerous information about
%   how to alter the
%   standard document layout produced by \LaTeX{}. It will tell you how  to
%   change things such that the beautiful output of \LaTeX{}
%   turns ugly or stunning, depending on your abilities.
\item[제 6 장] 표준 문서 레이아웃을 변경하는, 약간 위험할 수도 있는 내용을 다룬다. \LaTeX 의 아름다운 출력물을 어떻게 하면 엉망으로 만들거나 (능력에 따라) 더 근사하게 바꿀 수 있는지를 알려준다.
% \end{description}
% \bigskip
\end{description}
\bigskip

% \noindent It is important to read the chapters in order---the book is
% not that big, after all. Be sure to carefully read the examples,
% because a lot of the information is in the
% examples placed throughout the book.
\noindent 각 장을 순서대로 읽는 것이 중요하다. 이 소책자는 분량이 얼마 되지 않는다.
특히 예제를 주의깊게 보아야 하는데 책자 전체에 걸쳐 나타나는 예제에 많은 중요한 정보가 담겨 있기 때문이다.

% \bigskip
\bigskip
% \noindent \LaTeX{} is available for most computers, from the PC and Mac to large
% UNIX and VMS systems. On many university computer clusters you will
% find that a \LaTeX{} installation is available, ready to use.
% Information on how to access
% the local \LaTeX{} installation should be provided in the \guide. If
% you have problems getting started, ask the person who gave you this
% booklet. The scope of this document is \emph{not} to tell you how to
% install and set up a \LaTeX{} system, but to teach you how to write
% your documents so that they can be processed by~\LaTeX{}.
\noindent \LaTeX 은 PC, 맥, 대형 유닉스와 VMS 시스템에 이르기까지 거의 모든 컴퓨터에서 이용가능하다. 
대학의 컴퓨터실에는 이미 \LaTeX 이 설치되어 있어서 즉시 사용할 수 있을 것이다.
현재 자신이 이용하는 시스템에 \LaTeX 이 설치되어 있는지 어떻게 사용하면 되는지 알고 싶으면 \guide 를 보라. 
잘 되지 않으면 이 책자를 읽으라고 권한 사람에게 문의하라.
이 책이 다루는 범위는 \LaTeX 을 통하여 문서를 작성하는 방법에 대해 알려주려는 것이지 \LaTeX{} 시스템을 설치하고 설정하는 문제를 설명하지 않는다.

% \bigskip
\bigskip

% \noindent If you need to get hold of any \LaTeX{} related material,
% have a look at one of the Comprehensive \TeX{} Archive Network
% (CTAN) sites. The homepage is at
% \url{http://www.ctan.org}.
\noindent \LaTeX{} 관련 자료가 필요하다면 Comprehensive \TeX{} Archive Network (CTAN) 사이트를 방문해보라. 홈페이지는 \url{http://www.ctan.org}이다.

% You will find other references to CTAN throughout the book, especially
% pointers to software and documents you might want to download. Instead
% of writing down complete URLs, I just wrote \texttt{CTAN:} followed by
% whatever location within the CTAN tree you should go to.
이 책자 전반에 걸쳐 CTAN에 대한 언급이 나온다. 특히 소프트웨어나 안내문서를 다운로드하도록 지시할 때 그렇다. 완전한 URL을 적는 대신 단지 \texttt{CTAN:}이라고 표시하고 CTAN 트리상의 위치를 표시하였다.

% If you want to run \LaTeX{} on your own computer, take a look at what
% is available from \CTAN|systems|.
자신의 컴퓨터에서 \LaTeX 을 실행하고자 한다면 적당한 설치 배포판을 \CTAN|systems|에서 찾아볼 수 있다. 이 책자의 부록 \textbf{\ref{appx:installation}}도 참고하라.

% \vspace{\stretch{1}}
\vspace{\stretch{1}}
% \noindent If you have ideas for something to be
% added, removed or altered in this document, please let me know. I am
% especially interested in feedback from \LaTeX{} novices about which
% bits of this intro are easy to understand and which could be explained
% better.
\noindent 이 문서에 추가하거나 삭제 또는 변경해야 할 부분에 대한 의견이 있으면 저자에게 알려주기 바란다. 특히 이 안내서의 내용이 이해하기 쉬운지 더 좋은 설명 방법은 없을지에 대하여 \LaTeX{} 초보자로부터의 제안을 환영한다.

% \bigskip
\bigskip
% \begin{verse}
% \contrib{Tobias Oetiker}{tobi@oetiker.ch}%
% \noindent{OETIKER+PARTNER AG\\Aarweg 15\\4600 Olten\\Switzerland}
% \end{verse}
% \vspace{\stretch{1}}
\begin{verse}
\contrib{Tobias Oetiker}{tobi@oetiker.ch}%
\noindent{OETIKER+PARTNER AG\\Aarweg 15\\4600 Olten\\Switzerland}
\end{verse}
\vspace{\stretch{1}}

% \noindent The current version of this document is available on\\
% \CTAN|info/lshort|
\noindent 이 책자의 최신 버전은 \CTAN|info/lshort|에서 받아볼 수 있다.
그 하위 폴더 \CTAN|info/lshort/korean|에서 한국어판을 발견할 수 있을 것이다.

% \endinput



% %

% % Local Variables:
% % TeX-master: "lshort2e"
% % mode: latex
% % mode: flyspell
% % End:
