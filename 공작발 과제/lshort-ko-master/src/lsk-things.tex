%%%%%%%%%%%%%%%%%%%%%%%%%%%%%%%%%%%%%%%%%%%%%%%%%%%%%%%%%%%%%%%%%
% Contents: Things you need to know
% $Id$
%%%%%%%%%%%%%%%%%%%%%%%%%%%%%%%%%%%%%%%%%%%%%%%%%%%%%%%%%%%%%%%%%

% \chapter{Things You Need to Know}
\chapter{알아두어야 할 기본 사항}
% \begin{intro}
% The first part of this chapter presents a short
% overview of the philosophy and history of \LaTeXe. The second part
% focuses on the basic structures of a \LaTeX{} document.
% After reading this chapter, you should have a rough knowledge
% of how \LaTeX{} works, which you will need to understand the rest
% of this book.
% \end{intro}
\begin{intro}
이 장의 첫 부분에서 \LaTeXe의 철학과 역사에 대하여 간단히 개관한다.
그 다음 부분은 \LaTeX{} 문서의 기본 구조에 초점을 맞추었다.
이 장을 읽고 나면 \LaTeX 이 동작하는 방식에 대하여 어렴풋이 이해하게 될 것이며
이 책의 나머지 부분도 읽고 이해하고자 하는 필요성을 느끼게 될 것이다.
\end{intro}

% \section{A Bit of History}
% \subsection{\TeX}
\section{간략한 역사}
\subsection{\protect\TeX}

% \TeX{} is a computer program created by \index{Knuth, Donald E.}Donald
% E. Knuth \cite{texbook}. It is aimed at typesetting text and
% mathematical formulae. Knuth started writing the \TeX{} typesetting
% engine in 1977 to explore the potential of the digital printing
% equipment that was beginning to infiltrate the publishing industry at
% that time, especially in the hope that he could reverse the trend of
% deteriorating typographical quality that he saw affecting his own
% books and articles. \TeX{} as we use it today was released in 1982,
% with some slight enhancements added in 1989 to better support 8-bit
% characters and multiple languages. \TeX{} is renowned for being
% extremely stable, for running on many different kinds of computers,
% and for being virtually bug free. The version number of \TeX{} is
% converging to $\pi$ and is now at $3.141592653$.
\TeX 은 Donald E.~Knuth\index{Knuth, Donald E.}가 만든
컴퓨터 프로그램으로서 텍스트와 수학식을 조판하는 것이 그 목적이다\cite{texbook}. Knuth는 1977년부터 \TeX{} 조판 엔진을 
작성하였는데, 그 무렵 출판 산업에 스며들기 시작하던 디지털 인쇄 장비의
잠재적 가능성을 타진해보려는 의도와, 특히 자신의 책과 논문에서 그가 목도하고 있던 바 타이포그래피적 품질의
저하 경향을 반전시키고자 하는 희망을 가지고 시작한 일이었다.
현재 우리가 사용하는 \TeX 은 1982년에 발표되었고 1989년에 8비트 문자와 다국어 지원을 위한 약간의
개선을 거쳤다.
\TeX 은 고도로 안정적인 프로그램으로서 여러 종류의 컴퓨터에서 실행되며 거의 버그가 없는 것으로 유명하다.
\TeX 의 버전 번호는 $\pi$에 수렴하는데 현재는 $3.141592653$이다.\trfnote{%
  \TeX{} 프로그램의 버전은 \TeX\,Live 2021부터 3.141592653이다. 이 책의 영문판에는
  3.141592653이라고 하였다가 나중에 3.14159265로 고쳤는데 결과적으로 고치기 전의
  것이 옳은 상황이 된 것은 2021년이다.
}

% \TeX{} is pronounced ``Tech,'' with a ``ch'' as in the German word
% ``Ach''\footnote{In german there are actually two pronunciations for ``ch''
% and one might assume that the soft ``ch'' sound from ``Pech'' would be a
% more appropriate. Asked about this, Knuth wrote in the German Wikipedia:
% \emph{I do not get angry when people pronounce \TeX{} in their favorite way
% \ldots{} and in Germany many use a soft ch because the X follows the vowel
% e, not the harder ch that follows the vowel a. In Russia, `tex' is a very
% common word, pronounced `tyekh'. But I believe the most proper pronunciation
% is heard in Greece, where you have the harsher ch of ach and Loch.}}
% or in the Scottish ``Loch.'' The ``ch'' originates from the Greek
% alphabet where X is the letter ``ch'' or ``chi''. \TeX{} is also the first syllable
% of the Greek word technique. In an ASCII environment, \TeX{}
% becomes \texttt{TeX}.
\TeX 은 ``테흐(tech)''라고 발음한다.\trfnote{%
  우리나라나 영미권에서는 ``텍'' 또는 ``테크''\texttt{[tek]}로 발음하는 경우가 많다.
}
여기 ``ch''는 독일어 ``Ach''\footnote{%
  독일어에는 사실 두 가지 ``ch'' 발음이 있다. ``Pech''의 부드러운 ``ch'' 소리가 더 적절하지
  않으냐고 생각하는 사람도 있다. Knuth에게 이것에 대해 질문하였는데 그는 독일 위키백과에 다음과 같이 썼다.
  \emph{나는 사람들이 \TeX 을 제각기 좋을 대로 발음한다 해도 화나지 않는다 \ldots X가 모음 e 다음에 오기
  때문에 독일 사람들은 a 다음에 이어지는 거친 ch 소리가 아니라 부드러운 ch로 소리내는 사람이 많은데,
  러시아에서 `tex'은 익숙한 단어이고 `tyekh'로 발음한다. 그러나 가장 적절한 발음은 그리스어에 있다고 생각하며
  거기서는 ach와 Loch의 거친 ch 소리를 들을 수 있다.}}
또는 스코틀랜드어 ``Loch''에서
나는 소리와 같다. ``ch''는 그리스어의 ``chi''(`카이' 또는 `키')라고 하는 글자 $\chi$에서 온 것이다.
또 \TeX 은 기술(technique)을 뜻하는 그리스 단어의 첫 음절에서 따온 것이다.
아스키로 써야할 적에 \TeX 을 \texttt{TeX}으로 적는다.

% \subsection{\LaTeX}
\subsection{\LaTeX}

% \LaTeX{} enables authors to typeset and print their work at the highest
% typographical quality, using a predefined, professional layout. \LaTeX{} was
% originally written by \index{Lamport, Leslie}Leslie Lamport~\cite{manual}.
% It uses the \TeX{} formatter as its typesetting engine. These days \LaTeX{}
% is maintained by \index{The \LaTeX{} Project}the \LaTeX{} Project.
\LaTeX 은 미리 정의된 전문가 수준의 레이아웃을 사용하여 글쓰는 사람이 자신의 노작을 
최고의 타이포그래피 품질로 조판하고 인쇄할 수 있도록 도와준다.
\LaTeX 은 원래 Leslie Lamport\index{Lamport, Leslie}가 만든 것이었다\cite{manual}.
\LaTeX 은 \TeX 을 조판 엔진으로 사용한다. 현재 \LaTeX 은 \LaTeX{} 프로젝트 팀\index{LaTeX 프로젝트 팀@\LaTeX{} 프로젝트 팀}이
유지하고 있다.

% %In 1994 the \LaTeX{} package was updated by the \index{LaTeX3@\LaTeX
% %  3}\LaTeX 3 team, led by \index{Mittelbach, Frank}Frank Mittelbach,
% %to include some long-requested improvements, and to re\-unify all the
% %patched versions which had cropped up since the release of
% %\index{LaTeX 2.09@\LaTeX{} 2.09}\LaTeX{} 2.09 some years earlier. To
% %distinguish the new version from the old, it is called \index{LaTeX
% %2e@\LaTeXe}\LaTeXe. This documentation deals with \LaTeXe. These days you
% %might be hard pressed to find the venerable \LaTeX{} 2.09 installed
% %anywhere.

% \LaTeX{} is pronounced ``Lay-tech'' or ``Lah-tech.'' If you refer to
% \LaTeX{} in an ASCII environment, you type \texttt{LaTeX}.
% \LaTeXe{} is pronounced ``Lay-tech two e'' and typed \texttt{LaTeX2e}.
\LaTeX 은 ``레이텍'' 또는 ``라텍''으로 발음한다. 아스키로 써야할 적에 \texttt{LaTeX}으로 적는다.
\LaTeXe 는 ``레이텍 투 이''라고 읽으며 \texttt{LaTeX2e}로 적는다.

% %Figure~\ref{components} above % on page \pageref{components}
% %shows how \TeX{} and \LaTeXe{} work together. This figure is taken from
% %\texttt{wots.tex} by Kees van der Laan.

% %\begin{figure}[btp]
% %\begin{lined}{0.8\textwidth}
% %\begin{center}
% %\input{kees.fig}
% %\end{center}
% %\end{lined}
% %\caption{Components of a \TeX{} System.} \label{components}
% %\end{figure}

% \section{Basics}
\section{기초}

% \subsection{Author, Book Designer, and Typesetter}
\subsection{저자, 북 디자이너, 타입세터}

% To publish something, authors give their typed manuscript to a
% publishing company. One of their book designers then
% decides the layout of the document (column width, fonts, space before
% and after headings,~\ldots). The book designer writes his instructions
% into the manuscript and then gives it to a typesetter, who typesets the
% book according to these instructions.
어떤 것을 출판하기 위해서 저자는 출판사에 타자친 원고를 넘겨준다.
그러면 북 디자이너가 책의 레이아웃(문단 폭, 폰트, 표제부 전후의 간격 등)을 결정한다.
북 디자이너는 자신의 지시사항을 원고에 적어넣어서 타입세터(typesetter)에게 넘긴다.
타입세터는 이러한 지시사항에 따라 책을 조판한다.

% A human book designer tries to find out what the author had in mind
% while writing the manuscript. He decides on chapter headings,
% citations, examples, formulae, etc.\ based on his professional
% knowledge and from the contents of the manuscript.
북 디자이너는 사람인지라 저자가 원고를 쓸 때 어떤 생각으로 쓴 것인지 파악할 수 있다.
자신의 전문적 지식과 원고의 내용에 기초하여 어떤 것이 장의 표제이고 인용이거나 예문인지 혹은 수식인지 등을 판단한다.

% In a \LaTeX{} environment, \LaTeX{} takes the role of the book
% designer and uses \TeX{} as its typesetter. But \LaTeX{} is ``only'' a
% program and therefore needs more guidance. The author has to provide
% additional information to describe the logical structure of his
% work. This information is written into the text as ``\LaTeX{}
% commands.''
\LaTeX 으로 작업하는 환경에서 \LaTeX 은 북 디자이너의 역할을 맡고 \TeX 이 타입세터가 된다.
그러나 \LaTeX 은 단지 ``프로그램일 뿐''이다. 그러므로 가르쳐줘야 할 게 많다.
저자가 자신의 저작의 논리적 구조를 기술하는 추가적인 정보를 제공하여야 한다.
이러한 정보를 ``\LaTeX{} 명령어''라는 형태로 본문 속에 써넣는다.

% This is quite different from the \wi{WYSIWYG}\footnote{What you see is
%   what you get.} approach that most modern word processors, such as
% \emph{MS Word} or \emph{LibreOffice}, take. With these
% applications, authors specify the document layout interactively while
% typing text into the computer. They can see on the
% screen how the final work will look when it is printed.
이것은 \emph{MS Word}나 \emph{LibreOffice} 등 요즘 대부분의 워드 프로세서가 취하는 \wi{WYSIWYG}\footnote{What you see is what you get.}과는
완전히 다른 접근방법이다. 워드 프로세서에서 저자는 컴퓨터에 텍스트를 써넣으면서 동시에 문서 외양(레이아웃)을 눈으로 보면서 조절한다.
인쇄하였을 때 결과물의 모양을 화면으로 보고 있는 것이다.

% When using \LaTeX{} it is not normally possible to see the final output
% while typing the text, but the final output can be previewed on the
% screen after processing the file with \LaTeX. Then corrections can be
% made before actually sending the document to the printer.
\LaTeX 으로 작업한다면 내용을 적어넣는 작업 화면이 최종 출력물의 모양으로 보이지는 않는다.
그러나 \LaTeX 으로 파일을 처리한 후에 화면으로 최종 출력물의 모양을 미리보기할 수 있다.
문서를 프린터로 전송하여 실제 출력하기 전에 미리보기를 통해 확인하고 수정하는 것이 가능하다.

% \subsection{Layout Design}
\subsection{레이아웃 디자인}

% Typographical design is a craft. Unskilled authors often commit
% serious formatting errors by assuming that book design is mostly a
% question of aesthetics---``If a document looks good artistically,
% it is well designed.'' But as a document has to be read and not hung
% up in a picture gallery, the readability and understandability is
% much more important than the beautiful look of it.
% Examples:
% \begin{itemize}
% \item The font size and the numbering of headings have to be chosen to make
%   the structure of chapters and sections clear to the reader.
% \item The line length has to be short enough not to strain
%   the eyes of the reader, while long enough to fill the page
%   beautifully.
% \end{itemize}
타이포그래피 디자인은 전문분야이다. 비전문가 저자들이 북 디자인이라는 것을 예쁘게만 만들면
되는 거---``예술적으로 멋진 문서가 디자인이 잘 된 것''---라고 생각하기 때문에
심각한 서식 오류를 저지르는 수가 있다. 그러나 책이란 읽혀질 것이지 회랑에 내걸려 있을 것이 
아니므로 가독성과 이해가능성은 아름다운 외관보다 훨씬 중요하다.
예를 들면
\begin{itemize} \firmlist
  \item 장절 표제의 글자 크기와 번호매김은 독자들이 장절 편제를 통하여 문서의 구조를 명확히 파악할 수 있도록 선택되어야 한다.
  \item 글줄 길이는 독자의 눈에 부담주지 않을 정도로 짧아야 하지만 판면을 아름답게 보이게 할 정도로 적당히 길어야 한다.
\end{itemize}

% With \wi{WYSIWYG} systems, authors often generate aesthetically
% pleasing documents with very little or inconsistent structure.
% \LaTeX{} prevents such formatting errors by forcing the author to
% declare the \emph{logical} structure of his document. \LaTeX{} then
% chooses the most suitable layout.
WYSIWYG 시스템은 저자로 하여금 예쁘지만 구조적 일관성이 거의 없는 문서를 생성하게 하는 경향이 있다.
\LaTeX 은 저자가 자신의 문서의 \emph{논리적} 구조를 선언하게 만듦으로써 그런 서식 오류가 발생하지 않게 한다.
가장 적절한 레이아웃을 \LaTeX 이 선택한다.

% \subsection{Advantages and Disadvantages}
\subsection{장점과 단점}

% When people from the \wi{WYSIWYG} world meet people who use \LaTeX{},
% they often discuss ``the \wi{advantages of \LaTeX{}} over a normal
% word processor'' or the opposite.  The best thing to do when such
% a discussion starts is to keep a low profile, since such discussions
% often get out of hand. But sometimes there is no escaping \ldots
\wi{WYSIWYG} 세계의 사람들과 \LaTeX{} 사용자가 서로 만나면 ``워드 프로세서보다 나은 \LaTeX{} 사용의 장점''이나
그 반대 주제로 토론이 일어나곤 한다. 이런 논란은 대부분 엉뚱한 데로 흐르기 일쑤라 최선의 방책은 가만히 있는 것이지만 
때로 피할 수 없는 때도 있는 법\hdots\hdots.

% \medskip\noindent So here is some ammunition. The main advantages
% of \LaTeX{} over normal word processors are the following:
\medskip\noindent 그래서 여기 실탄 몇 발을 마련해두려 한다.
일반 워드 프로세서에 비하여 \LaTeX 이 가진 장점은 다음과 같다.

% \begin{itemize}
\begin{itemize} \oblivoirlist
% \item Professionally crafted layouts are available, which make a
%   document really look as if ``printed.''
\item 전문가 수준의 레이아웃으로 문서가 마치 실제 출판물처럼 보이게 한다.
% \item The typesetting of mathematical formulae is supported in a
%   convenient way.
\item 수학식의 조판과 편리한 입력이 가능하다.
% \item Users only need to learn a few easy-to-understand commands
%   that specify the logical structure of a document. They almost never
%   need to tinker with the actual layout of the document.
\item 사용자는 기억하기 쉬운 문서의 논리 구조를 지시하는 명령 몇 가지를 익히면 된다. 이것으로 문서 레이아웃을 이리저리 끼워맞추는 일을 하지 않을 수 있다.
% \item Even complex structures such as footnotes, references, table of
%   contents, and bibliographies can be generated easily.
\item 각주, 교차참조, 목차, 문헌목록과 같은 복잡한 구조도 쉽게 생성할 수 있다.
% \item Free add-on packages exist for many typographical tasks not directly supported by basic
%   \LaTeX. For example, packages are
%   available to include \PSi{} graphics or to typeset
%   bibliographies conforming to exact standards. Many of these add-on
%   packages are described in \companion.
\item 기본 \LaTeX 만으로 수행하기 어려운 타이포그래피적 요구를 충족하는 추가 패키지가 존재하고 자유로이 이용할 수 있다. 예를 들면 \PSi{} 그래픽을 포함하거나 정확한 출판 표준에 맞는 문헌목록을 조판하도록 하는 패키지가 있다. \companion 에서 많은 추가 패키지에 대해 설명한다.
% \item \LaTeX{} encourages authors to write well-structured texts,
%   because this is how \LaTeX{} works---by specifying structure.
\item \LaTeX 은 저자로 하여금 잘 구조화된 텍스트를 쓰도록 유도한다. 구조를 명시하는 것, 그것이야말로 \LaTeX{}  작업의 기본이다.
% \item \TeX, the formatting engine of \LaTeXe, is highly portable and free.
%   Therefore the system runs on almost any hardware platform
%   available.
\item \LaTeXe 의 조판 엔진인 \TeX 은 이식성이 뛰어나며 자유 소프트웨어이다. 거의 모든 하드웨어 플랫폼에서도 실행된다.

% %
% % Add examples ...
% %
% \end{itemize}
\end{itemize}

% \medskip
\medskip

% \noindent\LaTeX{} also has some disadvantages, and I guess it's a bit
% difficult for me to find any sensible ones, though I am sure other people
% can tell you hundreds \texttt{;-)}
\noindent \LaTeX 에 단점도 있다. 나로서는 납득할 만한 것이 없지만 수백 가지씩 말하는 사람도 틀림없이 있다. \texttt{;-)}

% \begin{itemize}
\begin{itemize} \oblivoirlist
% \item \LaTeX{} does not work well for people who have sold their
%   souls \ldots
\item \LaTeX 은 생각할 머리가 품절인 사람과 잘 맞지 않는다든가\hdots\hdots.
% \item Although some parameters can be adjusted within a predefined
%   document layout, the design of a whole new layout is difficult and
%   takes a lot of time.\footnote{Rumour says that this is one of the
%     key elements that will be addressed in the upcoming \LaTeX 3
%     system.}\index{LaTeX3@\LaTeX 3}
\item 미리 정의된 문서 서식의 일부 파라미터를 수정하는 것이 가능하기는 한데 새로운 레이아웃을 디자인하는 것은 너무 어렵고 시간이 많이 걸린다.\footnote{%
  소문에 의하면 이 점이 앞으로 나올 \LaTeX 3 시스템의 주요 요소가 될 것이라 한다.\index{LaTeX3@\LaTeX3}
}
% \item It is very hard to write unstructured and disorganized documents.
\item 구조적\cntrdot 조직적이지 않은 즉흥적 문서를 작성하기가 매우 어렵다.
% \item Your hamster might, despite some encouraging first steps, never be
% able to fully grasp the concept of Logical Markup.
\item 처음에 약간 되는 듯해 보여도 결국 \emph{논리적 마크업}의 개념을 완전히 이해하는 것이 햄스터에게는 무리다.
% \end{itemize}
\end{itemize}

% \section{\LaTeX{} Input Files}
\section{\LaTeX{} 입력 파일}

% The input for \LaTeX{} is a plain text file. On Unix/Linux text files are
% pretty common. On windows, one would use Notepad to create a text file. It
% contains the text of the document, as well as the commands that tell
% \LaTeX{} how to typeset the text. If you are working with a \LaTeX{} IDE, it will contain a program for creating
% \LaTeX{} input files in text format.
\LaTeX 은 플레인 텍스트 파일로 입력한다. 유닉스나 리눅스 시스템에서 텍스트 파일은 보편적이다. 윈도우즈에서 
Notepad(메모장) 응용 프로그램으로 만들고 편집할 수 있다.
입력 파일은 텍스트와 그 텍스트를 조판할 방식을 \LaTeX 에게 알려주는 명령으로 이루어진다. 
\LaTeX{} IDE(통합작업환경)은 입력 파일을 텍스트 포맷으로 작성하는 프로그램을 포함한다.

% \subsection{Spaces}
\subsection{공백}

% ``Whitespace'' characters, such as blank or tab, are
% treated uniformly as ``\wi{space}'' by \LaTeX{}. \emph{Several
%   consecutive} \wi{whitespace} characters are treated as \emph{one}
% ``space.''  Whitespace at the start of a line is generally ignored, and
% a single line break is treated as ``whitespace.''
% \index{whitespace!at the start of a line}
\LaTeX 은 빈 칸이나 탭 문자 같은 ``화이트스페이스'' 문자들을 ``\wi{스페이스}[whitespace]'' 문자와 완전히 동일하게 다룬다.
\emph{여러 개의 이어지는 화이트스페이스 문자}는 \emph{한 개의 ``스페이스''}로 취급한다.
입력 행의 앞부분에 있는 화이트스페이스는 보통 무시된다. 그리고 행 끝의 줄나눔 문자 한 개도 ``화이트스페이스''로 본다.\index{whitespace!at the start of a line}

% An empty line between two lines of text defines the end of a
% paragraph. \emph{Several} empty lines are treated the same as
% \emph{one} empty line. The text below is an example. On the left hand
% side is the text from the input file, and on the right hand side is the
% formatted output.
두 행 사이에 빈 줄을 두는 것은 문단의 끝임을 의미한다. \emph{빈 줄 여러 개}를 두어도 \emph{빈 줄 하나}와 동일하게 취급한다.
다음 예제를 보라. 왼쪽에 입력 파일에 입력한 텍스트를 보이고 오른쪽에 그 출력 결과를 나타내었다.

% \begin{example}
% It does not matter whether you
% enter one or several     spaces
% after a word.

% An empty line starts a new
% paragraph.
% \end{example}
\begin{example}
It does not matter whether you
enter one or several     spaces
after a word.

An empty line starts a new
paragraph.
\end{example}

% \subsection{Special Characters}
\subsection{특별한 문자}

% The following symbols are \wi{reserved characters} that either have a
% special meaning under \LaTeX{} or are not available in all the fonts.
% If you enter them directly in your text, they will normally not print,
% but rather coerce \LaTeX{} to do things you did not intend.
% \begin{code}
% \verb.#  $  %  ^  &  _  {  }  ~  \ . %$
% \end{code}
다음 기호들은 \wi{예약 문자}[reserved characters]라 부르며 \LaTeX 에서 특별한 의미로 쓰이거나
폰트로 찍을 수 없는 것이다. 입력 텍스트에 직접 적어넣는다면 대개 인쇄되어 나오지 않을 것이며 
\LaTeX 에게 의도하지 않은 일을 시키는 것이 될 것이다.
\begin{code}
\verb.#  $  %  ^  &  _  {  }  ~  \ . %$
\end{code}

% As you will see, these characters can be used in your documents all
% the same by using a prefix backslash:
위에 나온 것처럼 이 글자를 문서에 인쇄되게 하려면 문자 앞에 백슬래시를 붙여서 입력해야 한다.

% \begin{example}
% \# \$ \% \^{} \& \_ \{ \} \~{}
% \textbackslash
% \end{example}
\begin{example}
\# \$ \% \^{} \& \_ \{ \} \~{}
\textbackslash
\end{example}

% The other symbols and many more can be printed with special commands
% in mathematical formulae or as accents. The backslash character
% \textbackslash{} can \emph{not} be entered by adding another backslash
% in front of it (\verb|\\|); this sequence is used for
% line breaking. Use the \ci{textbackslash} command instead.
수학식이나 액센트 붙은 문자 등 특별한 명령으로 입력해야 인쇄되는 부호가 많다. 
다른 것과 달리 백슬래시 문자 \bs 는 그 앞에 백슬래시를 하나 더 붙여서(\verb|\\|) 입력하면 안 된다.
백슬래시 두 개는 강제 줄나눔을 나타내는 명령으로 쓰이기 때문이다. \ci{textbackslash}라는 명령을 써야 한다.


% \subsection{\LaTeX{} Commands}
\subsection{\LaTeX{} 명령}

% \LaTeX{} \wi{commands} are case sensitive, and take one of the following
% two formats:
\LaTeX{} \wi{명령}[commands]은 대소문자를 구별한다. 그리고 다음 두 가지 형식 중 하나를 취한다.

% \begin{itemize}
\begin{itemize}
% \item They start with a \wi{backslash} \verb|\| and then have a name
%  consisting of letters only. Command names are terminated by a
%  space, a number or any other `non-letter.'
\item \wi{백슬래시}[backslash] \verb|\|로 시작하여 글자(letter)로만 이루어진 이름을 갖는 형태. 명령 이름은 스페이스나 숫자 또는 `글자 아닌 것'이 오면 끝난다.\trfnote{%
  여기서 말하는 `글자'와 `글자 아닌 것'에 대하여 부언한다. \TeX 은 입력되는 문자(토큰)를 몇 개의 범주(category)로 구분하여 처리한다. 그 가운데 ``letter''라고 부르는 범주가 있다. 즉 \TeX{} 매크로 명령의 이름은 ``letter'' 범주에 속하는 문자로만 이루어진다는 의미이다. ``letter'' 범주에 속하는 문자의 범위는 \TeX{} 엔진에 따라 달라지는데 전통적\cntrdot 표준적으로 오직 영문자 알파벳(아스키 문자)만이 속한다고 생각하면 된다. 숫자나 기호문자는 ``글자''의 범주에 들지 않는다. 이 번역본에서 ``글자''라는 말이 \TeX{} category를 의미할 적에는 ``\emph{글자(letter)}''와 같이 표시했다.
}
% \item They consist of a backslash and exactly one non-letter.
\item 백슬래시 다음에 딱 한 개의 글자 아닌 것으로 이루어진 형태.
% \item Many commands exist in a `starred variant' where a star is appended
% to the command name.
\item 명령 이름에 별표를 추가하여 ``별표 붙은 명령'' 형태가 되는 것이 많다.
% \end{itemize}
\end{itemize}

% %
% % \\* doesn't comply !
% %

% %
% % Can \3 be a valid command ? (jacoboni)
% %
% \label{whitespace}
% \hyphenation{white-spaces white-space}
% \LaTeX{} ignores whitespace after commands. If you want to get a
% \index{whitespace!after commands}space after a command, you have to
% put either an empty parameter \verb|{}| and a blank or a special spacing command after the
% command name. The empty parameter \verb|{}| stops \LaTeX{} from eating up all the white space after
% the command name.
\label{whitespace}\index{whitespace!after commands}
\hyphenation{white-spaces white-space}
\LaTeX 은 명령 뒤의 화이트스페이스를 무시한다. 명령 뒤에 스페이스를 두어야 할 때는 빈 인자 \verb|{}|를 붙이거나
특별한 스페이스 명령을 사용해야 한다. 빈 인자 \verb|{}|는 \LaTeX 이 명령 이름 직후의 스페이스를 잡아먹지 못하도록 만든다.

% \begin{example}
% New \TeX users may miss whitespaces
% after a command. % renders wrong
% Experienced \TeX{} users are
% \TeX perts, and know how to use
% whitespaces. % renders correct
% \end{example}
\begin{example}
New \TeX users may miss whitespaces
after a command. % renders wrong
Experienced \TeX{} users are
\TeX perts, and know how to use
whitespaces. % renders correct
\end{example}


% Some commands require a \wi{parameter}, which has to be given between
% \wi{curly braces} \verb|{ }| after the command name. Some commands take
% \wi{optional parameters}, which are inserted after the command name in
% \wi{square brackets}~\verb|[ ]|.
% \begin{code}
% \verb|\|\textit{command}\verb|[|\textit{optional parameter}\verb|]{|\textit{parameter}\verb|}|
% \end{code}
% The next examples use some \LaTeX{}
% commands. Don't worry about them; they will be explained later.
\wi{인자}[parameter](parameters)를 요구하는 명령이 있다. 인자는 명령 이름 뒤에 \wi{중괄호}[curly braces] \verb|{ }|로 묶어서 전달한다.
\wi{옵션 인자}[optional parameters](optional parameters)를 취하는 경우도 있는데 이것은 명령 이름 뒤에 \wi{대괄호}[square brackets] \verb|[ ]|에 묶어서 전달한다.
인자는 보통 명령 자체가 요구하는 것이므로 생략할 수 없지만 옵션 인자(선택적 인자)는 생략가능하다.
\begin{code}
\verb|\|\textit{command}\verb|[|\textit{optional parameter}\verb|]{|\textit{parameter}\verb|}|
\end{code}
\LaTeX{} 명령의 사용례를 다음 보기에서 보였다. 지금 무슨 명령인지 모르겠다고 걱정할 필요 없다. 나중에 다 설명한다.

% \begin{example}
% You can \textsl{lean} on me!
% \end{example}
% \begin{example}
% Please, start a new line
% right here!\newline
% Thank you!
% \end{example}
\begin{example}
You can \textsl{lean} on me!
\end{example}

\vspace{-.5\onelineskip}

\begin{example}
Please, start a new line
right here!\newline
Thank you!
\end{example}

% \subsection{Comments}
% \index{comments}
\subsection{주석(Comments)} \index{comments}

% When \LaTeX{} encounters a \verb|%| character while processing an input file,
% it ignores the rest of the present line, the line break, and all
% whitespace at the beginning of the next line.
\LaTeX 이 입력 파일을 처리하던 중에 \verb|%| 문자를 만나면 그 줄의 나머지 부분과 줄나눔 문자 그리고 다음 줄 시작 부분의 화이트스페이스를 
무시한다.

% This can be used to write notes into the input file, which will not show up
% in the printed version.
이를 이용하여 입력 파일에 주석이나 메모를 적어두는 데 사용할 수 있다. 이 부분은 출력되지 않는다.

한 문장을 여러 줄로 나누어 입력할 때 나누어지는 위치에 있는 개행 문자나 화이트스페이스를 
무시하도록 하는 데 \texttt{\%} 문자를 이용할 수도 있다.

% \begin{example}
% This is an % stupid
% % Better: instructive <----
% example: Supercal%
%               ifragilist%
%     icexpialidocious
% \end{example}
\begin{example}
This is an % stupid
% Better: instructive <----
example: Supercal%
              ifragilist%
    icexpialidocious
\end{example}

% The \texttt{\%} character can also be used to split long input lines where no
% whitespace or line breaks are allowed.

% For longer comments you could use the \ei{comment} environment
% provided by the \pai{verbatim} package. Add the
% line \verb|\usepackage{verbatim}| to the preamble of your document as
% explained below to use this command.
\pai{verbatim} 패키지가 제공하는 \ei{comment} 환경을 이용하여 더 긴 주석문을 작성할 수도 있다.
\verb|\usepackage{verbatim}|이라는 문장을 문서의 전처리부에 적고 다음 예제와 같이 이 명령을 사용한다.

% \begin{example}
% This is another
% \begin{comment}
% rather stupid,
% but helpful
% \end{comment}
% example for embedding
% comments in your document.
% \end{example}
\begin{example}
This is another
\begin{comment}
rather stupid,
but helpful
\end{comment}
example for embedding
comments in your document.
\end{example}

% Note that this won't work inside complex environments, like math for example.
이것은 예컨대 수학식같은 좀더 복잡한 환경 안에서는 동작하지 않을 수 있음을 알아두자.

% \section{Input File Structure}
% \label{sec:structure}
\section{입력 파일의 구조} \label{sec:structure}

% When \LaTeXe{} processes an input file, it expects it to follow a
% certain \wi{structure}. Thus every input file must start with the
% command
% \begin{code}
% \verb|\documentclass{...}|
% \end{code}
% This specifies what sort of document you intend to write. After that,
% add commands to influence the style of the whole
% document, or load \wi{package}s that add new
% features to the \LaTeX{} system. To load such a package you use the
% command
% \begin{code}
% \verb|\usepackage{...}|
% \end{code}
\LaTeXe 는 입력 파일을 제대로 처리하기 위해 그것이 일정한 \wi{구조}를 갖추고 있을 것을 요구한다.
그래서 입력 파일의 맨 처음에 다음 문장이 있어야 한다.
\begin{code}
\verb|\documentclass{...}|
\end{code}
이 문장은 작성하는 문서가 어떤 종류의 것인지를 지정하는 것이다. 
이 다음에 전체 문서의 형식에 영향을 주는 명령이나 
\LaTeX 에 새로운 기능을 추가하는 \wi{패키지}[package]를 로드하는 문장이 온다.
패키지를 로드하려면 
\begin{code}
\verb|\usepackage{...}|
\end{code}
\noindent 라고 쓴다.

% When all the setup work is done,\footnote{The area between \texttt{\bs
%     documentclass} and \texttt{\bs
%     begin$\mathtt{\{}$document$\mathtt{\}}$} is called the
%   \emph{\wi{preamble}}.} you start the body of the text with the
% command
설정 작업이 다 되면\footnote{%
  \texttt{\bs documentclass}와 \texttt{\bs begin$\mathtt{\{}$document$\mathtt{\}}$}
  사이의 영역을 \emph{전처리부(preamble)}라고 한다.\index{전처리부}\index{preamble}
  문서 전체에 대한 설정 작업이 이루어지는 곳이 전처리부이다.
}
이제 문서의 본문을 시작한다는 뜻으로 다음 명령을 준다.

% \begin{code}
% \verb|\begin{document}|
% \end{code}
\begin{code}
\verb|\begin{document}|
\end{code}

% Now you enter the text mixed with some useful \LaTeX{} commands.  At
% the end of the document you add the
% \begin{code}
% \verb|\end{document}|
% \end{code}
% command, which tells \LaTeX{} to call it a day. Anything that
% follows this command will be ignored by \LaTeX.
이 다음에 문서의 내용을 텍스트와 적당한 \LaTeX{} 명령을 함께 섞어서 작성한다.
문서의 끝에는 
\begin{code}
\verb|\end{document}|
\end{code}
\noindent 라는 명령을 두어야 하는데 이것은 \LaTeX 에게 작업의 끝임을 알려주는 역할을 한다.
이 명령 이후에 오는 내용은 어떤 것이든 다 무시된다.

% Figure~\ref{mini} shows the contents of a minimal \LaTeXe{} file. A
% slightly more complicated \wi{input file} is given in
% Figure~\ref{document}.
그림~\ref{mini}\이 가장 간단한 \LaTeXe{} 입력 파일을 보여주고 있다.
그림~\ref{document}\는 조금 더 복잡한 보기이다.

% \begin{figure}[!bp]
% \begin{lined}{6cm}
% \begin{verbatim}
% \documentclass{article}
% \begin{document}
% Small is beautiful.
% \end{document}
% \end{verbatim}
% \end{lined}
% \caption{A Minimal \LaTeX{} File.} \label{mini}
% \end{figure}

% \begin{figure}[!bp]
% \begin{lined}{10cm}
% \begin{verbatim}
% \documentclass[a4paper,11pt]{article}
% % define the title
% \author{H.~Partl}
% \title{Minimalism}
% \begin{document}
% % generates the title
% \maketitle
% % insert the table of contents
% \tableofcontents
% \section{Some Interesting Words}
% Well, and here begins my lovely article.
% \section{Good Bye World}
% \ldots{} and here it ends.
% \end{document}
% \end{verbatim}
% \end{lined}
% \caption[Example of a Realistic Journal Article.]{Example of a Realistic
% Journal Article. Note that all the commands you see in this example will be
% explained later in the introduction.} \label{document}

% \end{figure}

\begin{figure}[!p]
\begin{lined}{6cm}
\begin{verbatim}
\documentclass{article}
\begin{document}
Small is beautiful.
\end{document}
\end{verbatim}
\end{lined}
\caption{\LaTeX{} 파일의 최소 작성례} \label{mini}
\end{figure}

\begin{figure}[!p]
\begin{lined}{10cm}
\begin{verbatim}
\documentclass[a4paper,11pt]{article}
% define the title
\author{H.~Partl}
\title{Minimalism}
\begin{document}
% generates the title
\maketitle
% insert the table of contents
\tableofcontents
\section{Some Interesting Words}
Well, and here begins my lovely article.
\section{Good Bye World}
\ldots{} and here it ends.
\end{document}
\end{verbatim}
\end{lined}
\caption[실제 저널 논문의 예]{실제 저널 논문의 예. 이 예제에 나온 명령은 이 책의 나중에 모두 설명한다.}
\label{document}

\end{figure}

한국어 독자를 위하여 한국어-한글 문서의 최소 작성례를 그림~\ref{minikor}에 보인다.

\begin{figure}[!p]
\begin{lined}{10cm}
\begin{verbatim}
\documentclass[a4paper]{article}
\usepackage{kotex}
\author{저자명}
\title{최소 작성례}
\begin{document}
\maketitle
\tableofcontents
\section{서론}
우리는 라텍을 배우기 시작했다. 이제 첫 문서를
작성한다.
\section{결론}
일찌감치 끝낸다.
\end{document}
\end{verbatim}
\end{lined}
\caption[한글 문서의 예]{한글 문서의 최소 작성례. UTF-8 유니코드 인코딩으로 저장하여야 한다.}\label{minikor}

\end{figure}


% \section{A Typical Command Line Session}
\section{명령행 작업}

% I bet you must be dying to try out the neat small \LaTeX{} input file
% shown on page \pageref{mini}. Here is some help:
% \LaTeX{} itself comes without a GUI or
% fancy buttons to press. It is just a program that crunches away at your
% input file. Some \LaTeX{} installations feature a graphical front-end where
% there is a \LaTeX{} button to start compiling your input file. On other systems
% there might be some typing involved, so here is how to coax \LaTeX{} into
% compiling your input file on a text based system. Please note: this
% description assumes that a working \LaTeX{} installation already sits on
% your computer.\footnote{This is the case with most well groomed Unix
% Systems, and \ldots{} Real Men use Unix, so \ldots{} \texttt{;-)}}
\pageref{mini}페이지에 보인 간단한 \LaTeX{} 입력 파일을 처리하는 방법이 무척 궁금할 것이다.
그 방법을 알아보자.
\LaTeX{} 자체는 뭔가 누를 수 있는 멋진 버튼을 갖춘 GUI 프로그램이 아니다.
입력 파일을 처리하는 것말고는 보여주는 것이 없다.
일부 \LaTeX{} 작업환경을 제공하는 프론트엔드 프로그램에 컴파일 버튼이 달린 GUI가 제공되기도 하지만
명령행 인터페이스만을 갖춘 시스템도 있다. 여기서는 텍스트 기반 시스템에서 
\LaTeX 을 불러서 입력 파일을 처리하도록 하는 방법에 대해서 살펴보겠다.
다만, 아래 기술은 \LaTeX 이 컴퓨터에 이미 설치되어서 잘 작동하고 있는 경우를 가정한다.\footnote{%
  잘 관리되고 있는 유닉스 시스템에서라면 당연히 그러할 것이다. 그리고 참된 인간이라면 모름지기 유닉스를 사용하는 법\hdots\hdots{} \texttt{;-)}
}

% \begin{enumerate}

\begin{enumerate}
% \item

%   Edit/Create your \LaTeX{} input file. This file must be plain ASCII
%   text.  On Unix all the editors will create just that. On Windows you
%   might want to make sure that you save the file in ASCII or
%   \emph{Plain Text} format.  When picking a name for your file, make
%   sure it bears the extension \eei{.tex}.
\item \LaTeX{} 입력 파일을 만들고 편집한다. 반드시 플레인 텍스트 파일이어야 한다. 유닉스 시스템의 모든 편집기가 플레인 텍스트 파일을 만든다. 윈도우즈에서는 \emph{플레인 텍스트} 형식으로 저장하도록 해야 한다. 파일 이름을 선택하고 확장명으로 \eei{.tex}을 부여한다.\trfnote{%
  한국어 사용자를 위한 첨언. 한국어 윈도우즈 사용자는 파일이 UTF-8 인코딩으로 저장되도록 주의를 기울여야 한다. Notepad 앱을 
  사용한다면 저장시에 이를 선택할 수 있다. 레이텍 전용 편집기는 UTF-8로 저장되는 것이 기본이지만 편집기의 설정에서 UTF-8 저장이
  활성화되어 있는지를 확인하는 것이 좋다. ``ANSI 인코딩''으로 저장하면 안 된다. 리눅스나 유닉스는 시스템의 기본 언어가 UTF-8이라면
  크게 신경쓰지 않아도 된다. 그리고 \underline{파일 이름은 영문 아스키 문자만}으로 짓는 것이 좋다. 한글 이름은 피하도록 하라.
}

% \item

% Open a shell or cmd window, \texttt{cd} to the directory where your input file is located and run \LaTeX{} on your input file. If successful you will end up with a
% \texttt{.pdf} file. It may be necessary to run \LaTeX{} several times to get
% the table of contents and all internal references right. When your input
% file has a bug \LaTeX{} will tell you about it and stop processing your
% input file. Type \texttt{ctrl-D} to get back to the command line.
% \begin{lscommand}
% \verb+xelatex foo.tex+
% \end{lscommand}
\item 셸 또는 cmd 창을 연다. \texttt{cd} 명령으로 저장된 파일이 있는 디렉터리(폴더)로 찾아 들어가서 이 파일에 대하여 \LaTeX{} 명령을 실행한다. 실행하여야 하는 \LaTeX{} 명령은 \texttt{xelatex}이나 \texttt{lualatex} 또는 \texttt{pdflatex}이다.\trfnote{%
  2019년 현재 한국어 문서 작성을 위하여 권장되는 것은 \texttt{xelatex}이다.
}
성공적으로 실행이 이루어지면 파일이름이 같은 \texttt{.pdf} 파일을 얻을 수 있다. 목차나 교차참조 등을 처리하기 위해서 \LaTeX 을 두 번 이상 실행해야 할 수도 있다. 만약 입력 파일에 오류가 있으면 \LaTeX{} 프로그램이 이를 알려주면서 처리를 중단할 것이다. 이럴 때는 \texttt{ctrl-D}를 눌러 명령행으로 돌아간다.\trfnote{%
  오류는 이를 수정하여 다시 시도하여야 한다. 많은 오류가 사소한 오타에서 비롯되므로 이를 찾아 수정하면 된다. 오류의 종류와 그 해결 방법에 대해서는 이 책을 끝까지 읽으면 더 많이 알 수 있게 된다.
}
\begin{lscommand}
\verb+xelatex foo.tex+
\end{lscommand}

% \end{enumerate}
\end{enumerate}

% \section{The Layout of the Document}
\section{문서 레이아웃}

% \subsection {Document Classes}\label{sec:documentclass}
\subsection{문서 클래스}\label{sec:documentclass}

% The first information \LaTeX{} needs to know when processing an
% input file is the type of document the author wants to create. This
% is specified with the \ci{documentclass} command.
% \begin{lscommand}
% \ci{documentclass}\verb|[|\emph{options}\verb|]{|\emph{class}\verb|}|
% \end{lscommand}
% \noindent Here \emph{class} specifies the type of document to be created.
% Table~\ref{documentclasses} lists the document classes explained in
% this introduction. The \LaTeXe{} distribution provides additional
% classes for other documents, including letters and slides.  The
% \emph{\wi{option}s} parameter customizes the behavior of the document
% class. The options have to be separated by commas. The most common options for the standard document
% classes are listed in
% Table~\ref{options}.
\LaTeX 이 요구하는 첫 번째 정보는 작성자가 만들려고 하는 문서의 유형에 대한 것이다. \ci{documentclass} 명령으로 이를 지정하여 준다.
\begin{lscommand}
\ci{documentclass}\verb|[|\emph{options}\verb|]{|\emph{class}\verb|}|
\end{lscommand}
\noindent 여기서 \emph{class}라는 것은 만들어질 문서의 종류를 나타내는 것이다. 
표~\ref{documentclasses}에 기본적인 문서 클래스를 열거하였다. \LaTeX{} 표준 배포판에는 이밖에 letter와 slide라는 
클래스가 더 있으며 다른 목적의 문서 클래스도 많다.
\emph{option} 파라미터는 문서 클래스의 동작 방식을 사용자가 지정하는 값들이다.
옵션은 쉼표로 분리하여 열거해야 한다. 표준 문서 클래스에 적용할 수 있는 흔히 쓰이는 옵션을 표~\ref{options}에 보였다.

% \begin{table}[!bp]
% \caption{Document Classes.} \label{documentclasses}
% \begin{lined}{\textwidth}
% \begin{description}

% \item [\normalfont\texttt{article}] for articles in scientific journals, presentations,
%   short reports, program documentation, invitations, \ldots
%   \index{article class}
% \item [\normalfont\texttt{proc}] a class for proceedings based on the article class.
%   \index{proc class}
% \item [\normalfont\texttt{minimal}] is as small as it can get.
% It only sets a page size and a base font. It is mainly used for debugging
% purposes.
%   \index{minimal class}
% \item [\normalfont\texttt{report}] for longer reports containing several chapters, small
%   books, PhD theses, \ldots \index{report class}
% \item [\normalfont\texttt{book}] for real books \index{book class}
% \item [\normalfont\texttt{slides}] for slides. The class uses big sans serif
%   letters. You might want to consider using the Beamer class instead.
%         \index{slides class}
% \end{description}
% \end{lined}
% \end{table}

\begin{table}[!bp]
\caption{문서 클래스} \label{documentclasses}
\begin{lined}{\textwidth}
\begin{description}

\item [\normalfont\texttt{article}] 
  과학 학술지 논문, 발표자료, 짧은 보고서, 프로그램 문서, 초대장 등을 위한 클래스
  \index{article class}
\item [\normalfont\texttt{proc}] 
  article 클래스에 기반한 프로시딩용 클래스 
  \index{proc class}
\item [\normalfont\texttt{minimal}] 
  가장 기본적인 클래스. 페이지 크기와 기본 폰트만을 설정한다. 오류추적 등을 위해 주로 사용한다.
  \index{minimal class}
\item [\normalfont\texttt{report}] 
  장(chapter)을 포함하는 클래스. 긴 보고서, 소책자, 박사논문 등에 쓸 수 있다.
  \index{report class}
\item [\normalfont\texttt{book}]
  단행본 제작을 위한 클래스
  \index{book class}
\item [\normalfont\texttt{slides}]
  슬라이드용 클래스. 산세리프체 큰 글씨를 기본 글꼴로 한다. 실제 슬라이드 제작에는 
  이것보다 beamer 클래스를 더 많이 쓴다.
  \index{slides class}
\end{description}
\end{lined}
\end{table}

% \begin{table}[!bp]
% \caption{Document Class Options.} \label{options}
% \begin{lined}{\textwidth}
% \begin{flushleft}
% \begin{description}
% \item[\normalfont\texttt{10pt}, \texttt{11pt}, \texttt{12pt}] \quad Sets the size
%   of the main font in the document. If no option is specified,
%   \texttt{10pt} is assumed.  \index{document font size}\index{base
%     font size}
% \item[\normalfont\texttt{a4paper}, \texttt{letterpaper}, \ldots] \quad Defines
%   the paper size. The default size is \texttt{letterpaper}. Besides
%   that, \texttt{a5paper}, \texttt{b5paper}, \texttt{executivepaper},
%   and \texttt{legalpaper} can be specified.  \index{legal paper}
%   \index{paper size}\index{A4 paper}\index{letter paper} \index{A5
%     paper}\index{B5 paper}\index{executive paper}

% \item[\normalfont\texttt{fleqn}] \quad Typesets displayed formulae left-aligned
%   instead of centred.

% \item[\normalfont\texttt{leqno}] \quad Places the numbering of formulae on the
%   left hand side instead of the right.

% \item[\normalfont\texttt{titlepage}, \texttt{notitlepage}] \quad Specifies
%   whether a new page should be started after the \wi{document title}
%   or not. The \texttt{article} class does not start a new page by
%   default, while \texttt{report} and \texttt{book} do.  \index{title}

% \item[\normalfont\texttt{onecolumn}, \texttt{twocolumn}] \quad Instructs \LaTeX{} to typeset the
%   document in \wi{one column} or \wi{two column}s.

% \item[\normalfont\texttt{twoside, oneside}] \quad Specifies whether double or
%   single sided output should be generated. The classes
%   \texttt{article} and \texttt{report} are \wi{single sided} and the
%   \texttt{book} class is \wi{double sided} by default. Note that this
%   option concerns the style of the document only. The option
%   \texttt{twoside} does \emph{not} tell the printer you use that it
%   should actually make a two-sided printout.
% \item[\normalfont\texttt{landscape}] \quad Changes the layout of the document to print in landscape mode.
% \item[\normalfont\texttt{openright, openany}] \quad Makes chapters begin either
%   only on right hand pages or on the next page available. This does
%   not work with the \texttt{article} class, as it does not know about
%   chapters. The \texttt{report} class by default starts chapters on
%   the next page available and the \texttt{book} class starts them on
%   right hand pages.

% \end{description}
% \end{flushleft}
% \end{lined}
% \end{table}
\begin{table}[!bp]
\caption{문서 클래스 옵션} \label{options}
\begin{lined}{\textwidth}
%\begin{flushleft}
\begin{description}
\item[\normalfont\texttt{10pt}, \texttt{11pt}, \texttt{12pt}] \quad 
  문서 본문 폰트 크기를 설정한다. 옵션을 따로 주지 않으면 \texttt{10pt}.
  \index{document font size}\index{base font size}\index{문서 폰트 크기}\index{폰트 크기}
\item[\normalfont\texttt{a4paper}, \texttt{letterpaper}, \ldots] \quad 
  용지 크기를 설정한다. 기본값은 \texttt{letterpaper}이다. 이밖에 
  \texttt{a5paper}, \texttt{b5paper}, \texttt{executivepaper},
  \texttt{legalpaper}를 줄 수 있다.
  \index{legal paper}\index{용지 크기}
  \index{paper size}\index{A4 paper}\index{letter paper} \index{A5
    paper}\index{B5 paper}\index{executive paper}

\item[\normalfont\texttt{fleqn}] \quad 
  별행 수식을 왼쪽 정렬로 식자한다. 이 옵션을 주지 않으면 가운데 정렬한다.

\item[\normalfont\texttt{leqno}] \quad 
  수식에 붙는 번호를 왼쪽에 붙인다. 이 옵션을 주지 않으면 오른쪽에 수식 번호가 인쇄된다.

\item[\normalfont\texttt{titlepage}, \texttt{notitlepage}] \quad 
  문서 표지를 별도의 한 페이지로 만들고 내용을 새 페이지로 시작할 것인지 그러지 않을 것인지를 지정한다.
  \texttt{article} 클래스는 표지면을 별도로 만들지 않는 것이 기본값이며 \texttt{report}와
  \texttt{book}은 별도 페이지로 하는 것이 기본이다.
  \index{title}\index{표지}

\item[\normalfont\texttt{onecolumn}, \texttt{twocolumn}] \quad 
  \wi{1단}[one column]이나 \wi{2단}[two column] 조판을 선택한다.

\item[\normalfont\texttt{twoside, oneside}] \quad 
  \wi{단면문서}[single sided]인지 \wi{펼침면} 조판(double sided)인지를 지정한다. \texttt{article} 클래스와 \texttt{report}는
  단면이 기본이고 \texttt{book}은 펼침면 조판이 기본이다.
  이 옵션이 의미하는 바는 문서의 모양을 어떻게 만들 것이냐에 관한 것일 뿐이고 
  프린터에게 양면인쇄를 하라는 명령을 보내는 것은 아니라는 점을 알아두자.
\item[\normalfont\texttt{landscape}] \quad 
  가로가 긴 페이지 레이아웃(landscape)을 선택한다.
\item[\normalfont\texttt{openright, openany}] \quad 
  장(chapter)이 오른쪽 페이지(홀수면)에서 시작하게 할 것인지 홀짝수면의 구분 없이
  시작할 수 있게 할 것인지 선택한다. \texttt{openright}를 선택하면 이전 장의 마지막이 홀수면일 때
  그 다음에 짝수면 하나를 내용없이 채우고 새 chapter를 다음 홀수면에서 시작하게 된다.
  이것은 \texttt{article}에서는 동작하지 않는 옵션이다. \texttt{report}는 다음 페이지에서
  바로 새로운 장을 시작하는 것이 기본값이고 \texttt{book}은 항상 오른쪽 페이지에서 시작하는 것이
  기본값이다.

\end{description}
%\end{flushleft}
\end{lined}
\end{table}



% Example: An input file for a \LaTeX{} document could start with the
% line
% \begin{code}
% \ci{documentclass}\verb|[11pt,twoside,a4paper]{article}|
% \end{code}
% which instructs \LaTeX{} to typeset the document as an \emph{article}
% with a base font size of \emph{eleven points}, and to produce a
% layout suitable for \emph{double sided} printing on \emph{A4 paper}.
% \pagebreak[2]
지금까지 설명을 바탕으로 예를 들어 보자.
\begin{code}
\ci{documentclass}\verb|[11pt,twoside,a4paper]{article}|
\end{code}
이 명령이 지시하는 바는 ``\texttt{article} 클래스''의 문서를 작성하되, 기본 폰트 크기는 ``11포인트''로 하고 
``펼침면(양면) 조판'' 형식으로 ``A4 용지''에 맞추라는 것이다.

% \subsection{Packages}
% \index{package} 
\subsection{패키지} \index{packages}

% While writing your document, you will probably find
% that there are some areas where basic \LaTeX{} cannot solve your
% problem. If you want to include \wi{graphics}, \wi{coloured text} or
% source code from a file into your document, you need to enhance the
% capabilities of \LaTeX.  Such enhancements are called packages.
% Packages are activated with the
% \begin{lscommand}
% \ci{usepackage}\verb|[|\emph{options}\verb|]{|\emph{package}\verb|}|
% \end{lscommand}
% \noindent command, where \emph{package} is the name of the package and
% \emph{options} is a list of keywords that trigger special features in
% the package. The \ci{usepackage} command goes into the preamble of the
% document. See section \ref{sec:structure} for details.
문서를 작성하다보면 기본 \LaTeX 만으로는 해결하기 어려운 문제를 만날 수 있다.
예컨대 그래픽을 포함해야 한다든가 색깔있는 텍스트를 쓴다든가 소스 코드를 파일로부터 읽어서 문서에 넣는다든가
하는 경우에 기본 \LaTeX 의 기능을 확장해야 할 필요가 생긴다.
이러한 기능 확장은 패키지를 통하여 이루어진다.
\begin{lscommand}
\ci{usepackage}\verb|[|\emph{options}\verb|]{|\emph{package}\verb|}|
\end{lscommand}
\noindent 패키지를 활성화하는 명령은 이와 같다. 여기서 \emph{package}는 패키지의 이름이고
\emph{options}는 그 패키지의 특정 기능과 동작을 제어하기 위한 키워드의 목록이다.
\ci{usepackage} 명령은 전처리부에서만 쓸 수 있다. \ref{sec:structure}절을 참고하라.

% Some packages come with the \LaTeXe{} base distribution
% (See Table~\ref{packages}). Others are provided separately. You may
% find more information on the packages installed at your site in your
% \guide. The prime source for information about \LaTeX{} packages is \companion.
% It contains descriptions on hundreds of packages, along with
% information of how to write your own extensions to \LaTeXe.
\LaTeXe{} 기본 배포판에 포함되어 딸려오는 패키지가 몇 가지 있다. (표~\ref{packages}\를 보라.)
별도로 제공되는 것도 많다. 어떤 패키지가 자신의 시스템에 설치되어 있는지는 \guide 를 참고할 수 있다.\trfnote{%
  일괄 설치 배포판 \TeX\,Live에 어떤 패키지가 있는지 알아보려면 \url{https://ctan.org/pkg}에서 
  검색해볼 수 있다.
}
\LaTeX{} 패키지에 관한 주요한 정보는 \companion에서 얻을 수 있는데 수백 가지 패키지와 그것이 하는 일에 대하여 설명하고 있다.

% Modern \TeX{} distributions come with a large number of packages
% preinstalled. If you are working on a Unix system, use the command
% \texttt{texdoc} for accessing package documentation.
요즘 \TeX{} 배포판은 엄청난 수의 패키지를 한꺼번에 설치해준다. 각 패키지 문서를 참조하려면 \texttt{texdoc}
명령을 사용한다.

% \begin{table}[btp]
% \caption{Some of the Packages Distributed with \LaTeX.} \label{packages}
% \begin{lined}{\textwidth}
% \begin{description}
% \item[\normalfont\pai{doc}] Allows the documentation of \LaTeX{} programs.\\
%  Described in \texttt{doc.dtx}\footnote{This file should be installed
%    on your system, and you should be able to get a \texttt{dvi} file
%    by typing \texttt{latex doc.dtx} in any directory where you have
%    write permission. The same is true for all the
%    other files mentioned in this table.}  and in \companion.

% \item[\normalfont\pai{exscale}] Provides scaled versions of the
%   math extension  font.\\
%   Described in \texttt{ltexscale.dtx}.

% \item[\normalfont\pai{fontenc}] Specifies which \wi{font encoding}
%   \LaTeX{} should use.\\
%   Described in \texttt{ltoutenc.dtx}.

% \item[\normalfont\pai{ifthen}] Provides commands of the form\\
%   `if\ldots then do\ldots otherwise do\ldots.'\\ Described in
%   \texttt{ifthen.dtx} and \companion.

% \item[\normalfont\pai{latexsym}] To access the \LaTeX{} symbol
%   font, you should use the \texttt{latexsym} package. Described in
%   \texttt{latexsym.dtx} and in \companion.

% \item[\normalfont\pai{makeidx}] Provides commands for producing
%   indexes.  Described in section~\ref{sec:indexing} and in \companion.

% \item[\normalfont\pai{syntonly}] Processes a document without
%   typesetting it.

% \item[\normalfont\pai{inputenc}] Allows the specification of an
%   input encoding such as ASCII, ISO Latin-1, ISO Latin-2, 437/850 IBM
%   code pages,  Apple Macintosh, Next, ANSI-Windows or user-defined one.
%   Described in \texttt{inputenc.dtx}.
% \end{description}
% \end{lined}
% \end{table}


\begin{table}[btp]
\caption{\LaTeX{} 기본 배포 패키지} \label{packages}
\begin{lined}{\textwidth}
\begin{description}
\item[\normalfont\pai{doc}] 
   \LaTeX{} 프로그램의 문서화를 가능하게 하는 패키지이다. \texttt{doc.dtx}\footnote{%
    이 파일은 시스템에 당연히 설치되어 있으며 \texttt{latex doc.dtx}를 실행하면 
    쓰기 권한 있는 폴더 어디서나 \texttt{dvi} 파일을
    얻을 수 있다. 다른 패키지에 대해서도 마찬가지이다.
   }%
   와 \companion 에 상세 설명이 있다.

\item[\normalfont\pai{exscale}] 
   수학 기본 폰트의 스케일 버전을 제공한다. \texttt{ltexscale.dtx}에 설명이 있다.

\item[\normalfont\pai{fontenc}] 
   폰트 인코딩을 지정한다. \texttt{ltoutenc.dtx} 문서에 설명이 있다.

\item[\normalfont\pai{ifthen}] 
  `if\ldots then \ldots otherwise \ldots' 형식의 명령을 지원한다. \texttt{ifthen.dtx}와 \companion을 보라.

\item[\normalfont\pai{latexsym}] 
  \LaTeX{} 기호 폰트를 사용가능하게 한다. \texttt{latexsym.dtx}와 \companion을 보라.

\item[\normalfont\pai{makeidx}] 
  색인 작성을 위한 패키지이다. 이 책자의 \ref{sec:indexing}절과 \companion에 설명되어 있다.

\item[\normalfont\pai{syntonly}] 
   \LaTeX 이 문법 검사만 하고 실제 출력물을 생성하지 않게 한다.

\item[\normalfont\pai{inputenc}] 
   입력 인코딩을 설정하게 한다.\trfnote{%
    앞서 역자주에서 언급한 대로 오늘날 라텍 시스템은 UTF-8을 읽고 쓸 수 있기 때문에 
    입력 인코딩은 중요한 문제가 아니게 되었다. 그러나 여전히 레거시 텍의 활용 빈도가 높은 상황에서는 
    이 패키지가 중요한 역할을 한다.
   }
   \texttt{inputenc.dtx}에 설명이 있다.
\end{description}
\end{lined}
\end{table}



% \subsection{Page Styles}
\subsection{페이지 스타일}

% \LaTeX{} supports three predefined \wi{header}/\wi{footer}
% combinations---so-called \wi{page style}s. The \emph{style} parameter
% of the \index{page style!plain@\texttt{plain}}\index{plain@\texttt{plain}}
% \index{page style!headings@\texttt{headings}}\index{headings@texttt{headings}}
% \index{page style!empty@\texttt{empty}}\index{empty@\texttt{empty}}
% \begin{lscommand}
% \ci{pagestyle}\verb|{|\emph{style}\verb|}|
% \end{lscommand}
% \noindent command defines which one to use.
% Table~\ref{pagestyle}
% lists the predefined page styles.
\LaTeX 은 세 종류의 상하단 \wi{면주}\index{header}\index{footer} 형식을 제공한다. 
면주 형식을 \wi{페이지 스타일}[page style]이라고 부른다.
\begin{lscommand}
\ci{pagestyle}\verb|{|\emph{style}\verb|}|
\end{lscommand}
\noindent 여기서 \emph{style} 위치에 올 수 있는 인자는 \texttt{plain},\index{plain@\texttt{plain}}
\texttt{headings},\index{headings@texttt{headings}}
\texttt{empty}\index{empty@\texttt{empty}}
가운데 하나이다.
표~\ref{pagestyle}에 미리 정의된 페이지 스타일을 열거하였다.

% \begin{table}[!hbp]
% \caption{The Predefined Page Styles of \LaTeX.} \label{pagestyle}
% \begin{lined}{\textwidth}
% \begin{description}

% \item[\normalfont\texttt{plain}] prints the page numbers on the bottom
%   of the page, in the middle of the footer. This is the default page
%   style.

% \item[\normalfont\texttt{headings}] prints the current chapter heading
%   and the page number in the header on each page, while the footer
%   remains empty.  (This is the style used in this document)
% \item[\normalfont\texttt{empty}] sets both the header and the footer
%   to be empty.

% \end{description}
% \end{lined}
% \end{table}

\begin{table}[!hbp]
\caption{\LaTeX 의 페이지 스타일} \label{pagestyle}
\begin{lined}{\textwidth}
\begin{description}

\item[\normalfont\texttt{plain}] 
  페이지 하단부 중앙에 페이지 번호를 인쇄한다. 기본 페이지 스타일이다.

\item[\normalfont\texttt{headings}] 
  각 페이지 상단에 페이지 번호와 장 표제를 인쇄하고 하단부는 비운다. (이 책자에서 사용하고 있는 스타일과 비슷하다.)
 
\item[\normalfont\texttt{empty}] 
  페이지의 상단과 하단을 모두 비우고 면주에 아무 것도 없게 한다.

\end{description}
\end{lined}
\end{table}


% It is possible to change the page style of the current page
% with the command
% \begin{lscommand}
% \ci{thispagestyle}\verb|{|\emph{style}\verb|}|
% \end{lscommand}
% A description how to create your own
% headers and footers can be found in \companion{} and in section~\ref{sec:fancy} on page~\pageref{sec:fancy}.
현재 페이지의 페이지 스타일을 바꾸려면
\begin{lscommand}
\ci{thispagestyle}\verb|{|\emph{style}\verb|}|
\end{lscommand}
\noindent 이라고 명령한다.
스스로 페이지의 상하단 면주를 설계하여 새로 만들 수도 있는데
그 방법을 \companion 과 이 책자 \pageref{sec:fancy}페이지 \ref{sec:fancy}절에서 찾을 수 있다.
% %
% % Pointer to the Fancy headings Package description !
% %

% \section{Files You Might Encounter}
\section{파일과 확장명}

% When you work with \LaTeX{} you will soon find yourself in a maze of
% files with various \wi{extension}s and probably no clue. The following
% list explains the various \wi{file types} you might encounter when
% working with \TeX{}. Please note that this table does not claim to be
% a complete list of extensions, but if you find one missing that you
% think is important, please drop me a line.
\LaTeX{} 작업을 하다보면 마주치게 되는 문제로 여러 가지 \wi{확장명}[extension]을 가진 
파일들이 생겨나는데 그게 무엇이고 왜 만들어졌는지 알 수 없다는 점이 있다.
다음에 보이는 목록은 \TeX{} 작업 중에 볼 수 있는 파일 종류를 설명한다.
이 목록에 있는 것이 모든 확장명 파일을 다 설명하고 있지는 않다. 중요한 확장명 파일에 대한 설명이
빠져 있다고 생각하면 저자에게 알려주기 바란다.

% \begin{description}

% \item[\eei{.tex}] \LaTeX{} or \TeX{} input file. Can be compiled with
%   \texttt{latex}.
% \item[\eei{.sty}] \LaTeX{} Macro package. Load this
%   into your \LaTeX{} document using the \ci{usepackage} command.
% \item[\eei{.dtx}] Documented \TeX{}. This is the main distribution
%   format for \LaTeX{} style files. If you process a .dtx file you get
%   documented macro code of the \LaTeX{} package contained in the .dtx
%   file.
% \item[\eei{.ins}] The installer for the files contained in the
%   matching .dtx file. If you download a \LaTeX{} package from the net,
%   you will normally get a .dtx and a .ins file. Run \LaTeX{} on the
%   .ins file to unpack the .dtx file.
% \item[\eei{.cls}] Class files define what your document looks
%   like. They are selected with the \ci{documentclass} command.
% \item[\eei{.fd}] Font description file telling  \LaTeX{} about new fonts.
% \end{description}

\begin{description}

\item[\eei{.tex}] \LaTeX{} 또는 \TeX{} 입력 소스 파일. \TeX{} 문서의 기본 확장명이다.
\item[\eei{.sty}] \LaTeX{} 매크로 패키지.
  이 파일은 \ci{usepackage} 명령으로 문서 중에 로드할 수 있다.
\item[\eei{.dtx}] 
  \TeX{} 문서화 파일. \LaTeX{} 스타일 파일을 배포할 때에 이 형식으로 하는 경우가 많다.
  문서와 코드를 동시에 포함하고 있는 파일로서 이를
  컴파일하여 스타일 파일과 관련 문서를 얻을 수 있다.
\item[\eei{.ins}] 
  .dtx 파일에 포함되어 있는 파일을 풀어내기 위하여 필요한 지침을 적은 
  인스톨 보조 파일. 온라인으로부터 \LaTeX{} 패키지를 내려받았을 때 
  .dtx와 .ins로 이루어져 있다면 .ins에 \LaTeX 을 적용하여 
  .dtx로부터 필요한 파일을 풀어낸다.
\item[\eei{.cls}]
  클래스 파일. \ci{documentclass} 명령으로 지정할 수 있다.
\item[\eei{.fd}] 
  폰트 기술(description) 파일. 
\end{description}



% The following files are generated when you run \LaTeX{} on your input
% file:
다음은 \LaTeX 을 실행할 때 작업 폴더에 생성되는 파일들이다.

% \begin{description}
% \item[\eei{.dvi}] Device Independent File. This is the main result of a classical \LaTeX{}
%   compile run. Look at its content with a DVI previewer
%   program or send it to a printer with \texttt{dvips} or a
%   similar application. If you are using \hologo{pdfLaTeX} then you should not see any of these files.
% \item[\eei{.log}] Gives a detailed account of what happened during the
%   last compiler run.
% \item[\eei{.toc}] Stores all your section headers. It gets read in for the
%   next compiler run and is used to produce the table of contents.
% \item[\eei{.lof}] This is like .toc but for the list of figures.
% \item[\eei{.lot}] And again the same for the list of tables.
% \item[\eei{.aux}] Another file that transports information from one
%   compiler run to the next. Among other things, the .aux file is used
%   to store information associated with cross-references.
% \item[\eei{.idx}] If your document contains an index. \LaTeX{} stores all
%   the words that go into the index in this file. Process this file with
%   \texttt{makeindex}. Refer to section \ref{sec:indexing} on
%   page \pageref{sec:indexing} for more information on indexing.
% \item[\eei{.ind}] The processed .idx file, ready for inclusion into your
%   document on the next compile cycle.
% \item[\eei{.ilg}] Logfile telling what \texttt{makeindex} did.
% \end{description}

\begin{description}
\item[\eei{.dvi}] 
  `장치 독립(Device Independent)'에서 온 확장명으로서 전통적인 \LaTeX{} 컴파일 결과 생성되는 출력 파일이다.
  DVI 프리뷰어 프로그램으로 화면상으로 결과를 보거나 \texttt{dvips}와 같은 유틸리티 프로그램을 이용하여 
  프린터로 보내거나 한다. 그러나 \hologo{pdfLaTeX}이나 최근의 새로운 엔진을 적용한 \LaTeX 을 주로 쓰는
  요즘은 이 파일을 보기가 힘들어졌다. 
\item[\eei{.log}] 
  컴파일 과정에서 일어난 상세한 기록을 담은 로그 파일.
\item[\eei{.toc}] 
  장절 표제를 저장하는 파일. 다음 번 컴파일 때에 목차를 생성하기 위해 이 파일을 읽는다.
\item[\eei{.lof}] 
  .toc와 같은 종류의 것이며 그림의 목록을 담고 있다.
\item[\eei{.lot}] 
  마찬가지로 표의 목록을 담고 있다.
\item[\eei{.aux}] 
  컴파일할 적에 다음 번 컴파일을 위하여 전달해야 할 정보를 적어두는 파일. 특히 교차참조에 필요한 
  정보를 저장하고 있다.
\item[\eei{.idx}] 
  색인을 만들고 있다면 \LaTeX 은 모든 색인용 단어들을 이 파일에 저장한다.
  이 파일을 \texttt{makeindex} 프로그램으로 처리하여 색인을 만든다. 
  색인 만들기와 관련해서 \pageref{sec:indexing}페이지의 \ref{sec:indexing}절을 참고하라.
\item[\eei{.ind}] 
  .idx 파일을 처리하여 문서에 들어갈 색인을 담고 있는 파일. 다음 번 컴파일 때 포함된다.
\item[\eei{.ilg}] 
  \texttt{makeindex}를 실행한 로그 파일.
\end{description}


% % Package Info pointer
% %
% %



% %
% % Add Info on page-numbering, ...
% % \pagenumbering

% \section{Big Projects}
\section{큰 규모의 글쓰기}

% When working on big documents, you might want to split the input file
% into several parts. \LaTeX{} has two commands that help you to do
% that.
큰 문서를 작업할 때에 여러 부분으로 입력 파일을 나누어놓는 것이 좋다.
\LaTeX 은 이런 작업에 필요한 두 가지 명령을 제공한다.

% \begin{lscommand}
% \ci{include}\verb|{|\emph{filename}\verb|}|
% \end{lscommand}
% \noindent Use this command in the document body to insert the
% contents of another file named \emph{filename.tex}. Note that \LaTeX{}
% will start a new page
% before processing the material input from \emph{filename.tex}.
\begin{lscommand}
\ci{include}\verb|{|\emph{filename}\verb|}|
\end{lscommand}
\noindent 이 명령을 본문에 쓰면 \emph{filename.tex}\,이라는 이름의 외부 파일의 내용을 
그 위치에 삽입한다. \emph{filename.tex}\,에 포함된 내용을 처리하기 직전에 
페이지가 나누어진다는 사실을 기억하라.

% The second command can be used in the preamble. It allows you to
% instruct \LaTeX{} to only input some of the \verb|\include|d files.
% \begin{lscommand}
% \ci{includeonly}\verb|{|\emph{filename}\verb|,|\emph{filename}%
% \verb|,|\ldots\verb|}|
% \end{lscommand}
% After this command is executed in the preamble of the document, only
% \ci{include} commands for the filenames that are listed in the
% argument of the \ci{includeonly} command will be executed.
다음 명령은 전처리부에서 쓸 수 있다. 이 명령으로 지정된 파일에 대한 
\verb|\include| 명령이 본문에 나올 경우에 그것만을 포함하라는 의미이다.
\begin{lscommand}
\ci{includeonly}\verb|{|\emph{filename}\verb|,|\emph{filename}%
\verb|,|\ldots\verb|}|
\end{lscommand}
\noindent 전처리부에 이 명령이 놓여 있으면 여기에 열거된 파일에 대한 
\verb|\include| 명령만이 실행된다. 다른 파일을 불러들이는 \verb|\include| 명령이 더 있다고 
하더라도 무시된다. 

% The \ci{include} command starts typesetting the included text on a new
% page. This is helpful when you use \ci{includeonly}, because the
% page breaks will not move, even when some include files are omitted.
% Sometimes this might not be desirable. In this case, use the
% \begin{lscommand}
% \ci{input}\verb|{|\emph{filename}\verb|}|
% \end{lscommand}
% \noindent command. It simply includes the file specified.
% No flashy suits, no strings attached.
\ci{include} 명령은 외부 파일을 새로운 페이지를 열어서 포함시킨다. \ci{includeonly}를 
쓰는 경우에, 해당 파일이 존재하지 않는 경우라 하더라도 페이지가 나누어지는 위치가 변하지 않기 때문에 유용하다. 
그러나 가끔 페이지나눔 없이 파일을 포함하여야 할 때가 있다. 이럴 때는 다음 명령을 사용한다.
\begin{lscommand}
\ci{input}\verb|{|\emph{filename}\verb|}|
\end{lscommand}
\noindent 명령이 주어진 위치에 다른 조치 없이 해당 파일을 바로 포함한다.

% To make \LaTeX{} quickly check your document use the \pai{syntonly}
% package. This makes \LaTeX{} skim through your document only checking for
% proper syntax and usage of the commands, but doesn't produce any (pdf) output.
% As \LaTeX{} runs faster in this mode you may save yourself valuable time.
% Usage is very simple:
\pai{syntonly} 패키지를 이용하여 문서를 빠르게 검토할 수 있다. 문서에서 명령의 사용법이 올바른지 
구문 오류는 없는지 체크하기만 하고 출력물 (pdf) 파일을 만들지 않는다.
이 검토는 매우 빠르게 이루어지므로 귀중한 시간을 절약할 수 있게 해주는데 특히 터미널로 문서를 작성하는 경우에 유용하다.

% \begin{verbatim}
% \usepackage{syntonly}
% \syntaxonly
% \end{verbatim}
% When you want to produce pages, just comment out the second line
% (by adding a percent sign).
\begin{verbatim}
\usepackage{syntonly}
\syntaxonly
\end{verbatim}
출력물을 얻으려면 두 번째 줄을 주석처리(줄 앞에 퍼센트 기호 \%를 추가)하면 된다.

% %

% % Local Variables:
% % TeX-master: "lshort2e"
% % mode: latex
% % mode: flyspell
% % End:
